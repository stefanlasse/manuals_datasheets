\section{Remote Programming}
\subsection{Communication Settings}
In order to remote control the \productName ~from a computer, it needs to be connected to the computer via the USB cable. The \productName ~will show up as a new Virtual COM Port (VCP). In some cases it might be necessary to install some drivers. They can be found here:\\
\url{http://www.ftdichip.com/Drivers/VCP.htm}\\
Once the Virtual COM Port has been installed successfully, the \productName ~can be controlled by sending commands via a serial terminal. A list of all available commands can be found in section~\ref{section_instruction_set}. The serial terminal needs to be configured as follows:
\begin{itemize}
\item 57600 Baud
\item 8 bit character size
\item no parity bit
\item 1 stop bit
\item no flow control
\end{itemize}

\subsection{Instruction set}
\label{section_instruction_set}
\lstMakeShortInline[language=sh,basicstyle=\ttfamily]@

The \productName ~has the following commands which can be used. Note, that the command parser is case sensitive. The command parameters, denoted by @<xxx>@, must be separated by either @SPACE@ or ``,'' or ``;'' or @TAB@. The command is completed by sending a  Carriage Return + Line Feed (@CRLF@) or Line Feed (@LF@).
\def \vdistace {3ex}
\vspace{\vdistace}

\begin{tabular}{ll}
Command: & @*RST@\\
Function: & Resets the \productName ~to the initial state.\\
Example: & @*RST@
\end{tabular}

\vspace{\vdistace}

\begin{tabular}{ll}
Command: & @*IDN?@\\
Function: & Returnes the identification name of the \productName.\\
Example: & @*IDN?@
\end{tabular}

\vspace{\vdistace}

\begin{tabular}{ll}
Command: & @*IDN <driver_idn>@\\
Function: & Sets the identification of the \productName.\\
Note: & The maximum number of characters is 20.\\
Example: & @*IDN NewName@\\
		& Sets the string ``NewName'' as IDN.
\end{tabular}

\vspace{\vdistace}

\begin{tabular}{ll}
Command: & @GETMOTSTATE <mot>@\\
Function: & Returns whether motor @<mot>@ is turned on or off.\\
Example: & @GETMOTSTATE 3@\\
		& Returns 1 if motor 3 is turned on or 0 if motor 3 is turned off.
\end{tabular}

\vspace{\vdistace}

\begin{tabular}{ll}
Command: & @MOVEABS <mot> <pos> <unit>@\\
Function: & Moves motor @<mot>@ to the absolute position @<pos> <unit>@.\\
		& The units can be steps, degree or radians.\\
Note: & The units must be written in lower case letters.\\
Example: & @MOVEABS 0 0 deg@\\
		& @MOVEABS 0 0 pi@\\
		& @MOVEABS 0 0 steps@\\
		& The examples do the same in different units.
\end{tabular}

\vspace{\vdistace}

\begin{tabular}{ll}
Command: & @MOVEREL <mot> <pos> <unit>@\\
Function: & Moves motor @<mot>@ relative to the current position.\\
		& The units can be steps, degree or radians.\\
Note: & The units must be written in lower case letters.\\
Example: & @MOVEREL 2 22.5 deg@\\
		& @MOVEREL 2 0.125 pi@\\
		& Both examples do the same in different units.
\end{tabular}

\vspace{\vdistace}

\begin{tabular}{ll}
Command: & @ZERORUN <mot>@\\
Function: & Finds the mechanical zero position of the motor.\\
Note: & During motor zero run no communication or usage\\
		& of the \productName ~is allowed.\\
Example: & @ZERORUN 1@\\
		& Finds the mechanical zero position of motor 1.
\end{tabular}

\vspace{\vdistace}

\begin{tabular}{ll}
Command: & @ENABLE <mot> <on/off>@\\
Function: & Turns motor @<mot>@ on (1) or off (0).\\
Note: & Both for enabeling and disabeling of a motor the same command is used.\\
Example: & @ENABLE 2 1@\\
		& Turns motor 2 on.\\
		& @ENABLE 2 0@\\
		& Turns motor 2 off.
\end{tabular}

\vspace{\vdistace}

\begin{tabular}{ll}
Command: & @GETPOS <mot> <unit>@\\
Function: & Returns the actual motor position in the given unit.\\
Example: & @GETPOS 1 deg@\\
		& Returns the current position of motor 1 in degree.
\end{tabular}

\vspace{\vdistace}

\begin{tabular}{ll}
Command: & @SAVECONF@\\
Function: & Saves all current configurations for all motors.\\
Note: & The driver configuration is stored in an EEPROM.\\
		& Maximum write cycles are 100000.\\
Example: & @SAVECONF@
\end{tabular}

\vspace{\vdistace}

\begin{tabular}{ll}
Command: & @LOADCONF@\\
Function: & Loads all saved configurations for all motors.\\
Example: & @LOADCONF@\\
\end{tabular}

\vspace{\vdistace}

\begin{tabular}{ll}
Command: & @GETOPTZEROPOS <mot>@\\
Function: & Returns the optical zero position of motor @<mot>@.\\
Example: & @GETOPTZEROPOS 3@\\
		& Returns the optical zero position of motor 3.
\end{tabular}

\vspace{\vdistace}

\begin{tabular}{ll}
Command: & @SETOPTZEROPOS <mot>@\\
Function: & Set the optical zero position for motor @<mot>@.\\
Note: & For the optical zero position the unit is always steps.\\
Example: & @SETOPTZEROPOS 3 574@\\
		& Sets the optical zero position of motor 3 to 574 steps.
\end{tabular}

\vspace{\vdistace}

\begin{tabular}{ll}
Command: & @GETWAITTIME <mot>@\\
Function: & Returns the wait time between two steps of a motor.\\
Example: & @GETWAITTIME 0@\\
		& Returns the wait time between two steps of motor 0.
\end{tabular}

\vspace{\vdistace}

\begin{tabular}{ll}
Command: & @SETWAITTIME <mot> <time>@\\
Function: & Sets the wait time between two steps to @<time>@ milliseconds\\
		& for motor @<mot>@.\\
Note: & The wait time must be an integer. The unit for the wait time\\
		& is always milliseconds.\\
Example: & @SETWAITTIME 1 5@\\
		& Sets the wait time of motor 1 to 5 milliseconds.
\end{tabular}

\vspace{\vdistace}

\begin{tabular}{ll}
Command: & @STOPALL@\\
Function: & Stops all motor movements immediately.\\
Example: & @STOPALL@\\
\end{tabular}

\vspace{\vdistace}

\begin{tabular}{ll}
Command: & @FACTORYRESET@\\
Function: & Resets the \productName ~to factory state.\\
Example: & @FACTORYRESET@\\
\end{tabular}

\vspace{\vdistace}

\begin{tabular}{ll}
Command: & @ISMOVING <mot>@\\
Function: & Returns the motor moving state.\\
		& 1: motor @<mot>@ is moving.\\
		& 0: motor @<mot>@ doesn't move.\\
Example: & @ISMOVING 0@\\
\end{tabular}

\vspace{\vdistace}

\begin{tabular}{ll}
Command: & @SETCONSTSPEED <mot> <dir> <time>@\\
Function: & Enables motor @<mot>@ to move infinite in direction @<dir>@. Possible\\
		& values for @<dir>@ are clock wise @CW@ or counter clock wise @CCW@.\\
		& @<time>@ specifies the time for one full rotation.\\
Example: & @SETCONSTSPEED 1 CW 10.0@\\
		& Moves motor 1 infinite in clockwise direction.\\
		& One full rotation takes 10 seconds.
\end{tabular}

\vspace{\vdistace}

\begin{tabular}{ll}
Command: & @SETFORBZONE <mot> <start> <stop>@\\
Function: & Defines a forbidden zone for motor @<mot>@. The motor will not move\\
		& into this zone. @<start>@ must be always smaller than @<stop>@. The unit\\
		& for @<start>@ and @<stop>@ is always steps.\\
Example: & @SETFORBZONE 0 148 1333@\\
		& Defines a forbidden zone for motor 0 between step 148 and step 1333.
\end{tabular}

\vspace{\vdistace}

\begin{tabular}{ll}
Command: & @ENABFORBZONE <mot> <val>@\\
Function: & Enables @<val=1>@ or disables @<val=0>@ the defined forbidden zone\\
		& for motor @<mot>@\\
Example: & @ENABFORBZONE 0 1@\\
		& Enables the forbidden zone for motor 0.\\
		& @ENABFORBZONE 3 0@\\
		& Disables the forbidden zone for motor 3.
\end{tabular}

\vspace{\vdistace}

\begin{tabular}{ll}
Command: & @SETPROGSTEP <step> <posMot0> <posMot1> <posMot2> <posMot3> <mode>@\\
Function: & Defines an internal program step for all four motors. @<step>@ is the program\\
		& sequence number. The position of all motors @<posMot0...4>@ must be given\\
		& in steps. The mode defines if the movement is to an absolute position @<mode=ABS>@\\
		& or if the movement is relative to the current motor position @<mode=REL>@.\\
Example: & @SETPROGSTEP 0 112 294 0 12 ABS@\\
		& Defines the internal program step 0 so that motor 0 moves to step 112,\\
		& motor 1 moves to step 294, motor 2 moves to step 0 and motor 3 moves to step 12.\\
		& @SETPROGSTEP 1 10 10 -10 -10 REL@\\
		& Defines the internal program step 1 so that motor 0 and motor 1 move\\
		& 10 steps forward relative to the actual position and motor 2 and 3 move\\
		& 10 steps backwards from the actual position.
\end{tabular}

\lstDeleteShortInline@

\newpage

