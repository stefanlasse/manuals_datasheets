
\section{Remote Programming}
\label{chp:remote_programming}
\subsection{Communication Settings}
\index{communication settings}
\index{COM port}
In order to remote control the \productName ~from a computer, connect it to a free USB port of the computer. The \productName ~will show up as a new Virtual COM Port (VCP). In some cases it might be necessary to install drivers, which can be found at
\begin{center}
  \url{http://www.lk-instruments.com/smc_software.html}
\end{center}
Once the Virtual COM Port has been installed successfully, the \productName ~can be controlled by sending commands via a serial terminal. A list of all available commands can be found in section~\ref{section_instruction_set}. The serial terminal needs to be configured as
\begin{itemize}
\item 57600 Baud
\item 8 bit character size
\item no parity bit
\item 1 stop bit
\item no flow control
\end{itemize}

\subsection{Instruction set}
\label{section_instruction_set}
\index{instruction set}
The following commands are available on the \productName .
Note, that the command parser is case sensitive. The command
parameters, denoted by \texttt{<xxx>}, must be separated by
either \texttt{SPACE} or ``,'' or ``;'' or \texttt{TAB}. The
command is completed by sending a  Carriage Return + Line Feed
(\texttt{CRLF}) or Line Feed (\texttt{LF}).\\
Note, that in order to simplify writing programs for the \productName , the remote interface uses a different numbering for the motors. On the device itself the motor channels are labeled with "Motor 1" to "Motor 4", but the remote interface starts counting at 0. Therefore in the following we will refer to the stepper motor connected to the first channel as "motor 0", to the motor connected to the second channel as "motor 1" and so on.

\clearpage

\def \vdistace {2ex}
%\vspace{\vdistace}

\begin{table}[h]
  \begin{tabularx}{\textwidth}{lX}
    Command:  & \texttt{*RST}\\
    Function: & Resets the \productName ~to the initial state and loads the last saved configuration.\\
    Example:  & \texttt{*RST}
  \end{tabularx}
\end{table}

\vspace{\vdistace}

\begin{table}[h]
  \begin{tabularx}{\textwidth}{lX}
    Command:  & \texttt{*IDN?}\\
    Function: & Returnes the identification name of the \productName.\\
    Example:  & \texttt{*IDN?} 
  \end{tabularx}
\end{table}

\vspace{\vdistace}

\begin{table}[h]
  \begin{tabularx}{\textwidth}{lX}
    Command:  & \texttt{GETMOTSTATE <mot>}\\
    Function: & Returns whether motor \texttt{<mot>} is turned on or off.\\
    Example:  & \texttt{GETMOTSTATE 3} \\
              & Returns 1 if motor 3 is turned on or 0 if motor 3 is turned off.
  \end{tabularx}
\end{table}

\vspace{\vdistace}

\begin{table}[h]
  \begin{tabularx}{\textwidth}{lX}
    Command:  & \texttt{ENABLE <mot> <on/off>}\\
    Function: & Turns motor \texttt{<mot>} on (1) or off (0).\\
    Note:     & Both for enabeling and disabeling of a motor the same command is used.\\
    Example:  & \texttt{ENABLE 2 1} \\
              & Turns motor 2 on. \\
              & \texttt{ENABLE 3 0} \\
              & Turns motor 3 off.
  \end{tabularx}
\end{table}

\vspace{\vdistace}

\begin{table}[h]
  \begin{tabularx}{\textwidth}{lX}
    Command:  & \texttt{MOVEABS <mot> <pos> <unit>}\\
    Function: & Moves motor \texttt{<mot>} to the absolute position \texttt{<pos> <unit>}.\\
              & The units can be steps, degree or radians.\\
    Note:     & The unit must be written in lower case letters.\\
    Example:  & \texttt{MOVEABS 1 0 deg} \\
              & \texttt{MOVEABS 1 0 pi} \\
              & \texttt{MOVEABS 1 0 steps} \\
              & The examples do the same in different units.
  \end{tabularx}
\end{table}

\vspace{\vdistace}

\begin{table}[h]
  \begin{tabularx}{\textwidth}{lX}
    Command:  & \texttt{MOVEREL <mot> <pos> <unit>}\\
    Function: & Moves motor \texttt{<mot>}relative to the current position.
                The units can be steps, degree or radians.\\
    Note:     & The unit must be written in lower case letters.\\
    Example:  & \texttt{MOVEREL 2 22.5 deg} \\
              & \texttt{MOVEREL 2 0.125 pi} \\
              & Both examples do the same in different units.
  \end{tabularx}
\end{table}

\vspace{\vdistace}

\begin{table}[h]
  \begin{tabularx}{\textwidth}{lX}
    Command:  & \texttt{ZERORUN <mot>}\\
    Function: & Finds the mechanical zero position of the motor.\\
    Note:     & During motor zero run no communication or usage of
                the \productName ~is allowed.\\
    Example:  & \texttt{ZERORUN 1} \\
              & Finds the mechanical zero position of motor 1.
  \end{tabularx}
\end{table}

\vspace{\vdistace}

\begin{table}[h]
  \begin{tabularx}{\textwidth}{lX}
    Command:  & \texttt{GETPOS <mot> <unit>}\\
    Function: & Returns the actual motor position in the given unit.\\
    Example:  & \texttt{GETPOS 1 deg} \\
              & Returns the current position of motor 1 in degree.
  \end{tabularx}
\end{table}

\vspace{\vdistace}

\begin{table}[h]
  \begin{tabularx}{\textwidth}{lX}
    Command:  & \texttt{ISMOVING <mot>}\\
    Function: & Returns the motor moving state.\\
              & 1: motor \texttt{<mot>} is moving.\\
		      & 0: motor \texttt{<mot>} is not moving.\\
    Example:  & \texttt{ISMOVING 0}\\
              & Returns 1 if motor 0 is moving and 0 if it is currently not moving.
  \end{tabularx}
\end{table}

\vspace{\vdistace}

\begin{table}[h]
  \begin{tabularx}{\textwidth}{lX}
    Command:  & \texttt{SAVECONF}\\
    Function: & Saves all current configurations for all motors.\\
    Note:     & The driver configuration is stored in an EEPROM.
		            Maximum write cycles are 100000.\\
    Example:  & \texttt{SAVECONF}
  \end{tabularx}
\end{table}

\vspace{\vdistace}

\begin{table}[h]
  \begin{tabularx}{\textwidth}{lX}
    Command:  & \texttt{LOADCONF}\\
    Function: & Loads all saved configurations for all motors.\\
    Example:  & \texttt{LOADCONF}
  \end{tabularx}
\end{table}

\vspace{\vdistace}

\begin{table}[h]
  \begin{tabularx}{\textwidth}{lX}
    Command:  & \texttt{GETZEROPOS <mot>}\\
    Function: & Returns the zero position of motor \texttt{<mot>}.\\
    Note:     & Zero positions are only available in steps.\\
    Example:  & \texttt{GETZEROPOS 3}\\
              & Returns the zero position of motor 3.
  \end{tabularx}
\end{table}

\vspace{\vdistace}

\begin{table}[h]
  \begin{tabularx}{\textwidth}{lX}
    Command:  & \texttt{SETZEROPOS <mot>}\\
    Function: & Set the zero position for motor \texttt{<mot>}.\\
    Note:     & For the zero position the unit is always steps.\\
    Example:  & \texttt{SETZEROPOS 3 574}\\
              & Sets the zero position of motor 3 to 574 steps.
  \end{tabularx}
\end{table}

\vspace{\vdistace}

\begin{table}[h]
  \begin{tabularx}{\textwidth}{lX}
    Command:  & \texttt{GETGEARRATIO <mot>}\\
    Function: & Returns the gear ratio of a motor.\\
    Example:  & \texttt{GETGEARRATIO 0}\\
              & Returns the gear ratio of motor 0.
  \end{tabularx}
\end{table}

\vspace{\vdistace}

\begin{table}[h]
  \begin{tabularx}{\textwidth}{lX}
    Command:  & \texttt{SETGEARRATIO <mot> <val>}\\
    Function: & Sets the gear ratio of motor \texttt{<mot>} to a gear ratio
                of \texttt{<val>}.\\
    Example:  & \texttt{SETGEARRATIO 2 3.0}\\
              & Sets the gear ratio of motor 2 to a value of 3.
  \end{tabularx}
\end{table}

\vspace{\vdistace}

\begin{table}[h]
  \begin{tabularx}{\textwidth}{lX}
    Command:  & \texttt{GETFULLROT <mot>}\\
    Function: & Returns the number of steps per full rotation without microsteps of a motor. Typical values for stepper motors are 200 or 400 steps per full rotation.\\
    Example:  & \texttt{GETFULLROT 0}\\
              & Returns the number of steps per full rotation of motor 0.
  \end{tabularx}
\end{table}

\vspace{\vdistace}

\begin{table}[h]
  \begin{tabularx}{\textwidth}{lX}
    Command:  & \texttt{SETFULLROT <mot> <val>}\\
    Function: & Sets the number of steps per full rotation of motor \texttt{<mot>} to \texttt{<val>} steps.\\
    Example:  & \texttt{SETFULLROT 1 400}\\
              & Sets the number of steps per full rotation of motor 1 to 400 steps.
  \end{tabularx}
\end{table}

\vspace{\vdistace}

\begin{table}[h]
  \begin{tabularx}{\textwidth}{lX}
    Command:  & \texttt{GETSUBSTEPS <mot>}\\
    Function: & Returns the number of substeps of a motor.\\
    Example:  & \texttt{GETSUBSTEPS 0}\\
              & Returns the number of substeps of motor 0.
  \end{tabularx}
\end{table}

\vspace{\vdistace}

\begin{table}[h]
  \begin{tabularx}{\textwidth}{lX}
    Command:  & \texttt{SETSUBSTEPS <mot> <val>}\\
    Function: & Sets the number of substeps of motor \texttt{<mot>} to \texttt{<val>} substeps.\\
    Note:     & Possible values for substeps are 1, 2, 4, 8, 16, 32.\\
    Example:  & \texttt{SETSUBSTEPS 1 4}\\
              & Sets the number of substeps of motor 1 to 4 substeps.
  \end{tabularx}
\end{table}

\vspace{\vdistace}

\begin{table}[h]
  \begin{tabularx}{\textwidth}{lX}
    Command:  & \texttt{GETCURR <mot>}\\
    Function: & Returns the motor current of a motor in ampere.\\
    Example:  & \texttt{GETCURR 0}\\
              & Returns the motor current of motor 0.
  \end{tabularx}
\end{table}

\vspace{\vdistace}

\begin{table}[h]
  \begin{tabularx}{\textwidth}{lX}
    Command:  & \texttt{SETCURR <mot> <val>}\\
    Function: & Sets the motor current of motor \texttt{<mot>} to \texttt{<val>} ampere.\\
    Note:     & The current can be adjusted between 0 and \unit[2.5]{A} with 8-bit resolution. The unit for the current is always ampere.\\
    Example:  & \texttt{SETCURR 0 1.0}\\
              & Sets the motor current of motor 0 to \unit[1]{A}.
  \end{tabularx}
\end{table}

\vspace{\vdistace}

\begin{table}[h]
  \begin{tabularx}{\textwidth}{lX}
    Command:  & \texttt{GETDECAY <mot>}\\
    Function: & Returns the setting of the decay mode of motor \texttt{<mot>}.\\
    Example:  & \texttt{GETDECAY 0}\\
              & Returns 0 if decay mode of motor 0 is set to slow, returns 1 if set to fast and returns 2 if set to mixed.
  \end{tabularx}
\end{table}

\vspace{\vdistace}

\begin{table}[h]
  \begin{tabularx}{\textwidth}{lX}
    Command:  & \texttt{SETDECAY <mot> <val>}\\
    Function: & Sets the decay mode of motor \texttt{<mot>} to slow \texttt{<val=0>}, fast \texttt{<val=1>} or mixed \texttt{<val=2>}.\\
    Note:     & For microstepping fast decay should be preferred. However if audible noise is an issue, one might consider using slow decay mode to reduce the noise.\\
    Example:  & \texttt{SETDECAY 1 0}\\
              & Sets decay mode of motor 1 to slow.\\
              & \texttt{SETDECAY 1 1}\\
              & Sets decay mode of motor 1 to fast.\\
              & \texttt{SETDECAY 1 2}\\
              & Sets decay mode of motor 1 to mixed.
  \end{tabularx}
\end{table}

\vspace{\vdistace}

\begin{table}[h]
  \begin{tabularx}{\textwidth}{lX}
    Command:  & \texttt{GETWAITTIME <mot>}\\
    Function: & Returns the wait time between two steps of a motor.\\
    Example:  & \texttt{GETWAITTIME 0}\\
              & Returns the wait time between two steps of motor 0.
  \end{tabularx}
\end{table}

\vspace{\vdistace}

\begin{table}[h]
  \begin{tabularx}{\textwidth}{lX}
    Command:  & \texttt{SETWAITTIME <mot> <time>}\\
    Function: & Sets the wait time between two steps to \texttt{<time>} milliseconds
                for motor \texttt{<mot>}\\
    Note:     & The wait time must be an integer. The unit for the wait time
                is always milliseconds.\\
    Example:  & \texttt{SETWAITTIME 1 5}\\
              & Sets the wait time of motor 1 to 5 milliseconds.
  \end{tabularx}
\end{table}

\vspace{\vdistace}

\begin{table}[h]
  \begin{tabularx}{\textwidth}{lX}
    Command:  & \texttt{STOPALL}\\
    Function: & Stops all motor movements immediately.\\
    Example:  & \texttt{STOPALL}
  \end{tabularx}
\end{table}

\vspace{\vdistace}

\begin{table}[h]
  \begin{tabularx}{\textwidth}{lX}
    Command:  & \texttt{SETCONSTSPEED <mot> <dir> <time>}\\
    Function: & Enables motor \texttt{<mot>} to move infinite in direction \texttt{<dir>}.
                Possible values for \texttt{<dir>} are clock wise \texttt{CW} or
                counter clock wise \texttt{CCW}. \\
    Note:     & That the wait time between two steps is overwritten in order to move with the desired speed.\\
    Example:  & \texttt{SETCONSTSPEED 1 CW 10.0}\\
              & Moves motor 1 infinite in clockwise direction.
                One full rotation takes 10 seconds.
  \end{tabularx}
\end{table}

\vspace{\vdistace}

\begin{table}[h]
  \begin{tabularx}{\textwidth}{lX}
    Command:  & \texttt{SETFORBZONE <mot> <start> <stop>}\\
    Function: & Defines a forbidden zone for motor \texttt{<mot>}.
                The motor will not move into this zone. \texttt{<start>} must
                be always smaller than \texttt{<stop>}. The unit for \texttt{<start>}
                and \texttt{<stop>} is always steps.\\
    Example:  & \texttt{SETFORBZONE 0 148 1333}\\
              & Defines a forbidden zone for motor 0 between step 148 and step 1333.
  \end{tabularx}
\end{table}

\vspace{\vdistace}

\begin{table}[h]
  \begin{tabularx}{\textwidth}{lX}
    Command:  & \texttt{ENABFORBZONE <mot> <val>}\\
    Function: & Enables \texttt{<val=1>} or disables \texttt{<val=0>} the
                defined forbidden zone for motor \texttt{<mot>}.\\
    Example:  & \texttt{ENABFORBZONE 0 1}\\
              & Enables the forbidden zone for motor 0.\\
              & \texttt{ENABFORBZONE 3 0}\\
              & Disables the forbidden zone for motor 3.
  \end{tabularx}
\end{table}

\vspace{\vdistace}

\begin{table}[h]
  \begin{tabularx}{\textwidth}{lX}
    Command:  & \texttt{SETPROGSTEP <step> <M0> <M1> <M2> <M3> <mode>}\\
    Function: & Defines an internal program step for all motors. \texttt{<step>}
                is the program sequence number. The position for all motors 
                \texttt{<M0...4>} must be given in steps. Mode defines if
                the movement is to an absolute position \texttt{<mode=ABS>}
                or a movement relative to the current position \texttt{<mode=REL>}.\\
    Example:  & \texttt{SETPROGSTEP 0 112 294 0 12 ABS}\\
              & Defines the program step 0. Motor 0 moves to 112,
                motor 1 moves to 294, motor 2 to 0 and motor 3 to 12.\\
              & \texttt{SETPROGSTEP 1 10 10 -10 -10 REL}\\
              & Defines the internal program step 1 so that motor 0 and motor 1 move
                10 steps forward and motor 2 and 3 move 10 steps backwards from the
                current position.
  \end{tabularx}
\end{table}

\vspace{\vdistace}

\begin{table}[h]
  \begin{tabularx}{\textwidth}{lX}
    Command:  & \texttt{FACTORYRESET}\\
    Function: & Resets the \productName ~to factory state.\\
    Example:  & \texttt{FACTORYRESET}
  \end{tabularx}
\end{table}

